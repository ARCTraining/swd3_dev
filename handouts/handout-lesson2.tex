\documentclass[8pt]{extarticle}
\usepackage{multicol}
\usepackage{listings}
\usepackage{fullpage}
\usepackage{tcolorbox}

\definecolor{navy}{RGB}{0,33,71}
\definecolor{lightgrey}{RGB}{235,238,238}
\definecolor{imperialblue}{RGB}{0,62,116}
\definecolor{lightblue}{RGB}{212,239,252}
\definecolor{coolgrey}{RGB}{157,157,157}
\setlength{\parindent}{0pt}

\newcommand{\boxed}[4][]{
  \begin{tcolorbox}[width=(0.95\textwidth)/#2,title=#3,before=,after=\hfill,
      colback=lightgrey,colframe=imperialblue,fonttitle=\bfseries,#1]
    #4
  \end{tcolorbox}
}
\newenvironment{navybox}[1][]{
  \begin{tcolorbox}[colback=lightgrey,colframe=navy,
      fonttitle=\bfseries,#1]
}{
  \end{tcolorbox}
}
\newenvironment{bluebox}[1][]{
  \begin{tcolorbox}[colback=lightgrey,colframe=imperialblue,
      fonttitle=\bfseries,#1]
}{
  \end{tcolorbox}
}

\newcommand{\furtherhelp}{
\subsection*{Further Help}
This course was developed by the Research Computing Service (RCS) of Imperial
College, in particular by the Research Software Engineering (RSE) team. The RSE
team are a part of Imperial ICT combining specialist knowledge in software
engineering with extensive experience in research. The team works with academic
groups on a wide range of projects whilst also organising community events and
training (such as this course) for the benefit of the research community. You
can find out more at the RSE team website (http://bit.ly/2SFropB) and the
Imperial RS Community website (http://bit.ly/2Hynf0q). You can also consult the
expertise of the RSE team at the weekly RCS clinic (http://bit.ly/2SENqbU).
\vspace{5mm}

You may also be interested in attending another course developed by the RSE team
titled \textbf{``Using Git to Code, Collaborate and Share''} bookable via the Graduate School (http://bit.ly/2SHIwuI).
}

%% \setlength{\itemsep}{-3pt}

\def\boxpad{2pt}
\tcbsetforeverylayer{left=\boxpad,right=\boxpad,top=\boxpad,bottom=\boxpad}

\begin{document}
\thispagestyle{empty}

\begin{center}
  {\LARGE
  Essential Software Engineering for Researchers - Lesson 2 - URL: v.gd/softeng}
\end{center}

\begin{navybox}[title=Why test?]
  \begin{multicols}{2}
    \begin{itemize}
    \itemsep-3pt
    \item Show that physical laws or mathematical relationships are correctly encoded
    \item Check that code works when running on a new system
    \item Make sure new code changes do not break existing functionality
    \item Ensure code correctly handles edge or corner cases
    \item Persuade others your code is reliable
    \item Check that code works with new or updated dependencies
    \end{itemize}
  \end{multicols}
\end{navybox}


\begin{bluebox}[title=Useful terms]
  \begin{itemize}
    \itemsep-3pt
  \item Unit testing - Validation of functionally independent components, in
    isolation
  \item Functional testing - Validation of a system of components
  \item Regression testing - Ensuring that new changes don't break existing
    tests, and that previously resolved issues stay fixed
  \item Test coverage - How comprehensive or ``complete is a suite of tests?
  \end{itemize}
\end{bluebox}


\begin{navybox}[title=Using pytest]
  A framework such as \textbf{pytest} simplifies the writing of tests and
  provides many powerful and convenient supporting functions.
  \begin{bluebox}[title=A simple example]
    \begin{lstlisting}[language=Python]
from temperature import to_fahrenheit

def test_answer():
    assert to_fahrenheit(30) == 86
    \end{lstlisting}
  \end{bluebox}

  \begin{bluebox}[title=Parameterisation]
    Run the same test multiple times using different input arguments. Use
    \lstinline{@pytest.mark.parametrize} to provide pairs of example inputs and
    expected outputs.
  \end{bluebox}
  \begin{bluebox}[title=Fixtures]
    Run different tests using the same input parameters. Useful if the input is
    slow to build or often reused. Use \lstinline{@pytest.fixture} with a
    function that returns the value to use as input.
  \end{bluebox}
  \begin{bluebox}[title=Tolerances]
    When comparing floating point numbers or a computed approximation it useful
    to be able to specify that values matching within a specified tolerance. Use
    \lstinline{pytest.approx} with a specified tolerance value.
  \end{bluebox}
  \begin{bluebox}[title=Turn your documentation into tests]
    Use the \textbf{doctest} module automatically turn your doc-strings into
    tests. Helps to test your code and keep your documentation up to date at the
    same time.
  \end{bluebox}
\end{navybox}


\furtherhelp

\end{document}
